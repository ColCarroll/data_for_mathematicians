\chapter{Before You Start}

\section{Before You Start}
We'll be moving mostly without a mouse. It is terrible at first, but pays dividends.
If you are using Ubuntu you can get to a terminal with \code{ctrl+alt+T}. On OSX,
use \code{cmd + space} to open spotlight, type in ``terminal'', then hit 
enter. I'll give some pointers to the most important commands to 
know in the terminal. Some important ones to start:

\begin{enumerate}
  \item \bash{ls} lists the files in the current directory
  \item \bash{cd directory} changes your directory to \code{directory}
    (if it exists).
  \item The up/down keys scroll through recent commands
  \item \code{tab} will autocomplete when it can. Sometimes you need to 
    hit it twice and it will give you options
  \item \code{ctrl+r} Gives you reverse search in the terminal, letting 
    you type in a few letters. Hitting \code{ctrl+r} more scrolls 
    backwards in these options, and \code{ctrl+g} cancel your search.
\end{enumerate}

\section{Setting up and navigating a development environment from the command 
  line}
If you can use an Ubuntu box, it will be much easier to set everything up, but 
it isn't terrible on OS X, either. In general, the better you can find your 
question on \href{http://stackoverflow.com/}{Stack Overflow}, the better you'll do
longterm, so check there early and often if you run into trouble.

We'll be using a few different terminals, and when you are meant to actively 
type something, I will try to copy the prompt: bash (my terminal) uses 
\bash{}, Python uses \py{} and R uses \R{}. Code with no prefix will be a 
\code{message printed back to you}.

Go to the appropriate section to set up your own environment for general 
purpose processing.
\subsection{Ubuntu}
We'll be using Ubuntu 12.04.  On Ubuntu, you get the 
\href{https://help.ubuntu.com/10.04/serverguide/apt-get.html}{Advanced Packaging Tool},
which we invoke with \bash{apt-get}.  We'll install a 
few other utilities off the bat:
\begin{description}
  \item[Get git] Start with\bash{sudo apt-get install git}.
Typing \bash{git} should now output something reasonable. Bad news if you see 
\code{-bash: git: command not found}, and you'll have to do some research to fix
this. 
  \item[Get python]Already built in -- I'd just use theirs.
  \item[Get R] We want 3.0.2, and for this we need to add their repository 
  before we can \code{apt-get} R. Check out 
  \href{http://cran.r-project.org/bin/linux/ubuntu/README}{this page} for 
  good information.
\end{description}

\subsection{OSX}
\begin{description}
  \item[Get Homebrew.] Instead of \code{apt-get}, we'll use 
  \href{http://brew.sh/}{Homebrew}. OS X ships with ruby, so you should be able 
  to copy/paste the command at the Homebrew website with no problem.
  \item[Get git] Now we can run
  \bash{brew install git}.  Typing \bash{git} should now output something 
  reasonable. Bad news if you see \code{-bash: git: command not found},
  and you'll have to do some \href{http://stackoverflow.com}{research}
  to fix this. 
  \item[Get python] Python is already built into OS X (open your terminal and 
  type \bash{python} to start hacking away), but you want a different version of
  Python.  Luckily, this is nice \bash{brew install python}.  This will install 
  Python 2.7.6, which is what we'll be using. 
  There are many discussions about Python 2 vs Python 3.  We follow the advice of
  \href{http://stackoverflow.com/questions/11938786/new-project-python-2-or-python-3}{this one},
  and use Python2 while writing code that will be Python3 compatible.  The main 
  differences are that division is no longer integer division (in python2, 
  \py{3/4 = 0}, in python3 \py{3/4 = 0.75} but \py{3//4 = 0}), more objects are
  \href{http://www.jeffknupp.com/blog/2013/04/07/improve-your-python-yield-and-generators-explained/}{generators},
  and the \code{print} function is invoked like a function. Python 3.4 was just 
  released, but \bash{brew install python3} will still install Python 3.3.3. Use
  at your own (minimal) risk.
  \item[Get R] As usual, \bash{brew install R} installs a good version of R 
  (I am on 3.0.2).
\end{description}

