\chapter{Python}

\subsection{Python Interpreter Hello World}
Open your terminal, type \bash{python} and then \py{print("Hello, World!")}.

That's a bit too easy. Let's instead run
\begin{lstlisting}[style=bash]
$ python
Python 2.7.5 (v2.7.5:ab05e7dd2788, May 13 2013, 13:18:45) 
[GCC 4.2.1 (Apple Inc. build 5666) (dot 3)] on darwin
Type "help", "copyright", "credits" or "license" for more information.
>>> x = "Hello, World!"
>>> print(x)
Hello, World!
>>> print(x + x)
Hello, World!Hello, World!
>>> print(5 * x)
Hello, World!Hello, World!Hello, World!Hello, World!Hello, World!
>>> print(5 * (x + "\n"))
Hello, World!
Hello, World!
Hello, World!
Hello, World!
Hello, World!
\end{lstlisting}

Notice that you can assign strings to a variable, and manipulate strings using 
addition in multiplication (and it works in a sane fashion).  The \code{\textbackslash n}
string is a newline.  The other useful character to know for now is 
\code{\textbackslash t}, which is a tab.
To quit the console, \code{ctrl+d} or \py{quit()}.  

\section{Cookbook}

\subsection{Serving files over a web server}\label{subsec:pyhttpserver}
This sounds intimidating, but you can literally go to any folder on your
machine and run \code{python -m SimpleHTTPServer 8080}.  Then if you open
a browser and navigate to \href{localhost:8080}{localhost:8080}, you will see 
all the files in your original folder.

\subsection{Serving a simple website with Flask, and Returning a 
list as JSON from a Flask app}\label{subsec:flask}
A great setup for simple websites is 
\href{http://flask.pocoo.org/docs/}{Flask} with bootstrap for 
css.  While working on an app to plot JSON data in 
\href{http://d3js.org/}{D3}, I used the following 

\begin{lstlisting}[style=script]
import random
from flask import Flask, Response
import json
app = Flask(__name__)

@app.route('/data')
def random_data():
  colors = ["red", "green", "blue"]
  data = [{
    "x": 10 * random.random(),
    "y": 10 * random.random(),
    "c": random.sample(colors, 1)[0],
    "size": random.randint(1,5)
    }
    for _ in range(10)]
  return Response(json.dumps(data), mimetype='application/json')

if __name__ == '__main__':
  app.run("127.0.0.1", port=5000, debug=True)
\end{lstlisting}

Now if I go to \code{127.0.0.1:5000/data}, I'll be served some json.  
\code{127.0.0.1} is another name for \code{localhost} (so 
\code{localhost:5000/data} will also do it).  Setting \code{debug=True}
lets me edit the script and it will auto-reload, as well as gives a
helpful error page with an interactive python terminal if something
goes wrong.  One other small point about this: if the \code{data}
object was a dictionary, I could have returned \code{jsonify(data)},
(where I would also need to \code{from flask import jsonify}), but 
\href{http://stackoverflow.com/questions/12435297/how-do-i-jsonify-a-list-in-flask}{for security reasons}
I need to return this one myself by converting it to json, and 
setting the proper mimetype.
In this script, I create an \code{app} object, define one route,
and serve json over that route.

\subsection{Printing a number with commas}

Check out \href{http://legacy.python.org/dev/peps/pep-3101/}{PEP 3101}
on new string formatting.  \code{"\$\{:,d\}".format(10 ** 9)} outputs
the string \code{'\$1,000,000,000'}.  The \code{\{\ldots\}} indicates that there
is something to be replaced, you can put a key (in this case \code{0}) before
the colon to decide what will be replaced, the \code{d} says the object is an
integer, and the \code{,} says to add commas every third digit.
