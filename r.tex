\chapter{R}
\subsection{R Interpreter Hello World}
Happily, we may again type \bash{R}, and \R{print("Hello, World!")}.
We can't run exactly the same session as we did in Python, since R doesn't let us
mix algebra and strings quite as nicely, but there are equivalents.

\begin{lstlisting}[style=bash]
$ R

R version 3.0.2 (2013-09-25) -- "Frisbee Sailing"
Copyright (C) 2013 The R Foundation for Statistical Computing
Platform: x86_64-apple-darwin10.8.0 (64-bit)

R is free software and comes with ABSOLUTELY NO WARRANTY.
You are welcome to redistribute it under certain conditions.
Type 'license()' or 'licence()' for distribution details.

  Natural language support but running in an English locale

R is a collaborative project with many contributors.
Type 'contributors()' for more information and
'citation()' on how to cite R or R packages in publications.

Type 'demo()' for some demos, 'help()' for on-line help, or
'help.start()' for an HTML browser interface to help.
Type 'q()' to quit R.

> x <- "Hello, World!"
> print(x)
[1] "Hello, World!"
> print(paste(x,x))
[1] "Hello, World! Hello, World!"
> print(paste(rep(x, 5), collapse=""))
[1] "Hello, World!Hello, World!Hello, World!Hello, World!Hello, World!"
> cat(paste(rep(x, 5), collapse="\n"))
Hello, World!
Hello, World!
Hello, World!
Hello, World!
Hello, World!> 
\end{lstlisting}
A few things to note about this:
\begin{enumerate}
\item We use the \code{paste} function to concatenate strings.
\item The \code{rep(x, 5)} created a vector with five copies of \code{x}. We 
  could have done this by hand with \R{c(x, x, x, x, x)}.  More on R vectors
  later.
\item The \code{paste} function accepts a \code{collapse} argument, which tells 
  it how to join a vector of strings.
\item In the last command, we used \code{cat} instead of \code{paste}.  The
  \code{paste} function would print ``\textbackslash n'', while the \code{cat} function
  interprets these as newlines.  
\item The \code{cat} function doesn't automatically put a newline after it 
  does its thing, which is why the interpreter starts typing after the last
  ``Hello, World!''
\end{enumerate}


\section{Tools for using R}

R has some fantastic support for getting up and working quickly and easily.  A 
few ways to help get hacking:

\begin{description}
\item[Defining variables.] R uses the symbol \code{<-} instead of \code{=}.  
  Using \code{=} will still work (see 
  \href{http://blog.revolutionanalytics.com/2008/12/use-equals-or-arrow-for-assignment.html}{this article}
  if you're interested), but it is typically reserved for setting function
  arguments.
\item[Using Packages]  You can import R code in a number of ways.  I just use
  \R{require(ggplot2)} to import, for example, ggplot2.  You could also use
  \R{library(ggplot2)}, with 
  \href{http://stackoverflow.com/questions/5595512/what-is-the-difference-between-require-and-library}{almost identical}
  results.  If I am importing a local file, use 
  \R{source('../relative/path/to/myfile.R')}.  Note that with this, you can run
  code, assign values to variables, and define functions.  A good workflow
  is to have an interactive console open, figure out how to make R do what
  you want to do, copy that over to a file, and \code{source} the file.
\item[Installing Packages.] A great strength of R is that it has some cutting
  edge packages available.  These are uploaded to CRAN, the ``Comprehensive
  R Archive Network'', which is ``a network of ftp and web servers around the 
  world that store identical, up-to-date, versions of code and documentation for
  R''.  The majority of packages you'll want are installed
  via \R{install.packages("ggplot2")}.  You'll be asked to choose a mirror, 
  and typically the best will be the 
  \href{http://blog.rstudio.org/2013/06/10/rstudio-cran-mirror/}{cloud mirror}
  that RStudio maintains on an Amazon EC2 instance -- number 0.  It syncs
  with the central CRAN mirror in Austria once a day, and is typically beats
  any other mirror for speed (in Austin, the cloud mirror is faster than using
  the mirror in Dallas). 

  Occasionally, you'll want something really cutting edge that you find on 
  github.  There's a nice package from 
  \href{had.co.nz}{Hadley Wickham} that makes this easy:

  \begin{lstlisting}[style=bash]
  > require(devtools)
  Loading required package: devtools
  > install_github('ggplot2')
  Installing github repo ggplot2/master from hadley
  Downloading ggplot2.zip from https://github.com/h...
  Installing package from /var/folders/k1/3qh0v1017...
  arguments 'minimized' and 'invisible' are for Win...
  Installing ggplot2
  '/Library/Frameworks/R.framework/Resources/bin/R'...
    '/private/var/folders/k1/3qh0v1017ql8p4xm7hqzv_...
    --library='/Users/colinc/Library/R/3.0/library'...

  * installing *source* package 'ggplot2' ...
  ** R
  ** data
  *** moving datasets to lazyload DB
  ** inst
  ** tests
  ** preparing package for lazy loading
  ** help
  *** installing help indices
  ** building package indices
  ** testing if installed package can be loaded
  * DONE (ggplot2)
  > require(ggplot2)
  Loading required package: ggplot2
  > sessionInfo()
  R version 3.0.2 (2013-09-25)
  Platform: x86_64-apple-darwin10.8.0 (64-bit)

  locale:
  [1] en_US.UTF-8/en_US.UTF-8/en_US.UTF-8/C/en_US.UTF-8/en_US.UTF-8

  attached base packages:
  [1] stats     graphics  grDevices utils    
  [5] datasets  methods   base     

  other attached packages:
  [1] ggplot2_0.9.3.1.99 devtools_1.4.1    

  loaded via a namespace (and not attached):
   [1] colorspace_1.2-4   dichromat_2.0-0   
   [3] digest_0.6.4       evaluate_0.5.1    
   [5] grid_3.0.2         gtable_0.1.2      
   [7] httr_0.2           labeling_0.2      
   [9] MASS_7.3-29        memoise_0.1       
  [11] munsell_0.4.2      parallel_3.0.2    
  [13] plyr_1.8           proto_0.3-10      
  [15] RColorBrewer_1.0-5 RCurl_1.95-4.1    
  [17] reshape2_1.2.2     scales_0.2.3      
  [19] stringr_0.6.2      tools_3.0.2       
  [21] whisker_0.3-2     
  > 
  \end{lstlisting}
  Note the use of \R{sessionInfo()} at the end to view all the attached
  packages and their versions.


\item[Getting help.] It is notoriously difficult to google for help with 
  R, given its name (though, really, who hasn't come up with unexpected 
  results while searching for help with \LaTeX, Python, or Ruby?  I've even
  heard a story of a trip to manpages.com in search of bash help.) The built-in
  R help pages are accessed via \code{?}, and are actually quite helpful.

  Note that \code{?} will only search packages which are currently loaded.  
  Try the following session:

  \begin{lstlisting}[style=bash]
  > ?geom_point
  No documentation for ‘geom_point’ in specified packages and libraries:
  you could try ‘??geom_point’
  > require(ggplot2)
  Loading required package: ggplot2
  > ?geom_point 
  \end{lstlisting}

  The suggestion to use \code{??} is also a good one: when I type it in, I see

  \begin{lstlisting}[style=bash]
  Help files with alias or concept or title
  matching ‘geom_point’ using regular
  expression matching:


  ggplot2::geom_point
                 Points, as for a scatterplot
  ggplot2::geom_pointrange
                 An interval represented by a
                 vertical line, with a point
                 in the middle.


  Type '?PKG::FOO' to inspect entries
  'PKG::FOO', or 'TYPE?PKG::FOO' for entries
  like 'PKG::FOO-TYPE'.




  (END) 
  \end{lstlisting}
  This command has searched through all \emph{installed}, not necessarily loaded,
  packages.  If you see a sweet function on a blog post and want to find which 
  package it is in, you need to either go to google,
  \href{http://www.rseek.org/}{RSeek},
  \href{http://crantastic.org/}{crantastic},
  or use the \code{sos} package:

  \begin{lstlisting}[style=bash]
  > install.packages('sos')
  Installing package into ‘/Users/colinc/Library/R/3.0/library’
  (as ‘lib’ is unspecified)
  --- Please select a CRAN mirror for use in this session ---
  trying URL 'http://cran.rstudio.com/bin/macosx/contrib/3.0/sos_1.3-8.tgz'
  Content type 'application/x-gzip' length 213172 bytes (208 Kb)
  opened URL
  ==================================================
  downloaded 208 Kb


  The downloaded binary packages are in
    /var/folders/k1/3qh0v1017ql8p4xm7hqzv_3m84fy03/T//Rtmp3Dpn1e/downloaded_packages
  > require(sos)
  Loading required package: sos
  Loading required package: brew

  Attaching package: ‘sos’

  The following object is masked from ‘package:utils’:

      ?

  > findFn('geom_point')
  found 197 matches;  retrieving 10 pages
  2 3 4 5 6 7 8 9 10 

  Downloaded 175 links in 50 packages.
  >
  \end{lstlisting}

This creates a local website with links to all the packages that you might find
\code{geom\_point} in.  

\end{description}


