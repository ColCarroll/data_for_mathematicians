\chapter{Web Development}
This covers dashboards and sharing data.
\section{Dictionary}
There are a lot of different technologies one might use.

\subsection{SASS}

``Syntactically Awesome Style Sheets'', \href{http://sass-lang.com/}{SASS} is an
extension of CSS3 that allows for things like variables and inheritance so that
you can write sane CSS.

\section{Cookbook}

\subsection{Installing NodeJS}\label{subsec:install_nodejs}
Either use the installer from \href{http://nodejs.org/}{the website}, or (on Ubuntu), run
\begin{lstlisting}[style=bash]
$ sudo apt-get install python-software-properties
$ sudo add-apt-repository ppa:chris-lea/node.js
$ sudo apt-get update
$ sudo apt-get install nodejs
\end{lstlisting}

\subsection{Using npm}
NodeJS uses \href{https://www.npmjs.org/}{npm} as its package manager
(hence, NodePackageManager). You might like a package manager because
different projects you work on depend on different versions of the same
packages.  If you want to install a package called \code{express} for a project,
you can do so with \code{npm install express}.  If you want all projects you
run on your computer to be able to use express, then add a \code{-g} flag
to install globally: \code{node install -g express}. 

\subsection{Starting an AngularJS App}

First, install nodejs (\autoref{subsec:install_nodejs}).
We use \href{http://yeoman.io/}{Yeoman} to scaffold an
\href{http://angularjs.org/}{AngularJS} app, and manage dependencies with 
\href{http://bower.io/}{Bower}.  We 
\begin{enumerate}
\item Install these tools with npm,
\item install a Yeoman generator that actually scaffolds the angular app,
\item make a directory,
\item scaffold the app,
\item create a git repo
\end{enumerate}

\begin{lstlisting}[style=bash]
$ npm install -g yo grunt-cli bower
npm http GET https://registry.npmjs.org/yo
npm http GET https://registry.npmjs.org/grunt-cli
npm http GET https://registry.npmjs.org/bower

(this continues)

$ npm install -g generator-angular
npm http GET https://registry.npmjs.org/generator-angular

(more noise)

$ mkdir sample_project
$ cd sample_project
$ yo angular
(interactive stuff, noise)
$ git init
Initialized empty Git repository in /Users/colinc/sample_project/.git/
$ git add --all
$ git commit -am "Initialized an empty angular repo"

(noise)
\end{lstlisting}

Now you have a bunch of folders -- 
\begin{description}
\item[\code{Gruntfile.js}] We use Grunt to manage packages (Grunt: the package
manager for the web), and the Gruntfile is where we set up how Grunt does its
job.
\item[\code{app/}] This is the folder where our app actually lives.  Being
javascript, it is a static app, so you could just drop in here and serve
it with a Python SimpleHTTPServer (\autoref{subsec:pyhttpserver}).  
\end{description}
