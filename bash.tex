\chapter{Bash}

\section{Cookbook}


\subsection{Count files in a folder.}

If you have a lot of files in a folder, \bash{ls -1} (that's the numeral 1)
will list the files in the folder, one to a line, without a header.  This
is piped to a word count of lines.


\bash{ls -1 | wc -l}


\subsection{View head of gzipped file}

Use \code{gzip}, with the \code{-c} flag to output to \code{stdout}, and 
\code{-d} to decompress (instead of compress).  This can be piped to head.

\bash{gzip -cd | head}


\subsection{Dealing with corrupt gzips}

Sometimes a process will be consuming a bunch of gzips and it will
fail for some reason (this happened to me with an
\code{unexpected end of file} error). To find the files that may fail,
run \bash{gzip -t}. To recover what part of the gzip you can recover,
\bash{gunzip < corrupted.gz > corrupted.partial}

\subsection{Search for code snippets}

If you have just a few files, \code{grep} can be a great choice, but
if you're searching through a bigger directory, and know what you're
looking for is in (say) a Python file, use 
\bash{grep -r -i --include \*.py "import re" *}
\code{-r} is recursive, so it looks in sub folders, \code{-i} is case
insensitive, \code{--include} says which files to search.
